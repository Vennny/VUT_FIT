\documentclass[a4paper,11pt]{article}
\usepackage[left=2cm,text={17cm, 24cm},top=3cm]{geometry}
\usepackage[utf8]{inputenc}
\usepackage[czech]{babel}
\usepackage{times}
\newcommand{\bibtex}{\textsc{Bib}\negthinspace\TeX}
\usepackage{etoolbox}
\apptocmd{\thebibliography}{\raggedright}{}{}

\begin{document}
\begin{titlepage}
\begin{center}
{\Huge\textsc{Vysoké učení technické v~Brně}}\\
\vspace{0.7em}{\huge\textsc{Fakulta informačních technologií}}\\
\vspace{\stretch{0.382}}
{\LARGE Typografie a publikování\,--\,3. projekt}\\
\Huge {Citace\\}
\vspace{\stretch{0.618}}
\end{center}
{\Large\today\hfill Václav Trampeška}
\end{titlepage}

\section{Co je to \TeX}
\TeX je program pro počítačovou sazbu dokumentů. Byl vytvořen v~roce 1978 americkým profesorem Stanfordovy univerzity Donaldem Ervinem Knuthem. Nástroj obsahuje přes 300 příkazů, díky kterým \TeX oproti svým konkurentům umožňuje kvalitnější a~konzistentí zapisování jak obyčejného textu, tak především matematických vzorců, algoritmů a~podobných struktur \cite{tex_online}. Další nespornou výhodou nástroje je to, že je multiplatformní \cite{clanek1}. Autor popsal používání nástroje \TeX také ve své knize \cite{tex_book}.

\section{Co je to \LaTeX}
\LaTeX je neznámější nadstavbou \TeX u. Byl vytvořen Lesliem Lamportonem v~roce 1984. \LaTeX je oproti \TeX u jednodušší na zápis a používání, přičemž neztrácí žádné jeho výhody. Myšlenkou \LaTeX u je usměrnit autory k tomu, aby kladli důraz hlavně na obsah dokumentu a~ne na jeho vzhled. \cite{latex_about} \LaTeX je vhodný hlavně na středně dlouhé a~dlouhé dokumenty. Není problém pomocí něho i~vytvořit prezentaci \cite{clanek2}. Využití \LaTeX u se rozšířilo natolik, že se v něm dnes běžně sázejí i knihy \cite{tex_online2} nebo lze také vykreslovat vektorovou grafiku \cite{serial}.

\section{Jak se naučit \LaTeX}
\LaTeX se zprvu může zdát mocným, ale složitým nástrojem. Pokud ale člověk upustí od předsudků a~nahlédne pod pokličku \LaTeX u zjistí, že vytvářet dokumenty vlastně není složité. Toto pomyslné náhlednutí nám mohou umožnit jak kvanta internetových návodů, tak i~knihy jako Latex: podrobný průvodce\cite{latex_pruvodce} dostupný v knihovně na VUT FIT. \LaTeX je i~skvelým nástrojem na psaní bakalářských prací. Některým studentům dokonce zalíbí natolik, že se stane i~samotným tématem práce \cite{prace1}\cite{prace2}.


\newpage
\renewcommand{\refname}{Reference}
\bibliography{bibliografie}
\bibliographystyle{czplain}
\end{document}