\documentclass[a4paper, 11pt, twocolumn]{article}

\usepackage[left=1.5cm, text={18cm, 25cm}, top=2.5cm]{geometry}
\usepackage[czech]{babel}
\usepackage[utf8]{inputenc}
\usepackage[IL2]{fontenc}

\usepackage{times}
\usepackage{amsmath}
\usepackage{amssymb}
\usepackage{amsthm}
\usepackage{stmaryrd}

\theoremstyle{definition}
\newtheorem{definice}{Definice}

\theoremstyle{definition}
\newtheorem{veta}{Věta}

\begin{document}
\begin{titlepage}
\begin{center}

{\Huge\textsc{Fakulta informačních technologií}\\
\vspace{0.4em}\textsc{Vysoké učení technické v~Brně}\\}
\vspace{\stretch{0.382}}
{\LARGE Typografie a publikování\,--\,2. projekt\\
\vspace{0.3em}Sazba dokumentů a matematických výrazů\\}
\vspace{\stretch{0.618}}
{\Large 2019\hfill Václav Trampeška(xtramp00)}
\end{center}
\end{titlepage}


\section*{Úvod}
V~této úloze si vyzkoušíme sazbu titulní strany, matematic\-kých vzorců, prostředí a~dalších textových struktur obvyklých pro technicky zaměřené texty (například rovnice (\ref{rovnice1}) nebo Definice \ref{definice1} na straně \pageref{definice1}). Pro odkazovaní na vzorce a~struktury zásadně používáme příkaz \verb|\label| a~\verb|\ref| případně \verb|\pageref| pokud se chceme odkázat na stranu výskytu.

Na titulní straně je využito sázení nadpisu podle optického středu s~využitím zlatého řezu. Tento postup byl probírán na~přednášce. Dále je použito odřádkování se zadanou relativní velikostí 0.4 em a~0.3 em.

\section{Matematický text}
Nejprve se podíváme na sázení matematických symbolů\linebreak a~výrazů v~plynulém textu včetně sazby definic a~vět s~využitím balíku \texttt{amsthm}. Rovněž použijeme poznámku pod čarou s~použitím příkazu \verb|\footnote|. Někdy je vhodné použít konstrukci \verb|\mbox{}|, která říká, že text nemá být zalomen.

\begin{definice} \label{definice1}
Zásobníkový automat \emph{(ZA) je definován jako sedmice tvaru $A = (Q,\Sigma,\Gamma,\delta,q_0,Z_0,F)$, kde:} 

\begin{itemize}
\item \emph{$Q$ je konečná množina} vnitřních (řídicích) stavů,
\item $\Sigma$ \emph{je konečná} vstupní abeceda,
\item $\Gamma$ \emph{je konečná} zásobníková abeceda,
\item $\delta$ \emph{je} přechodová funkce $Q \times$ ($\Sigma$ $\cup$\verb|{|$\epsilon$\verb|}|) $\times$ $\Gamma$ $\rightarrow$ $2^{Q \times \Gamma ^*}$,
\item $q_0$ $\in Q$ \emph{je} počáteční stav, $Z_0$ $\in$ $\Gamma$ \emph{je} startovací symbol zásobníku \emph{a F $\subseteq Q$ je množina }koncových stavů.
\end{itemize}

\mbox{Nechť $P = (Q,\Sigma,\Gamma,\delta,q_0,Z_0,F)$} je zásobníkový~auto\-mat. \emph{Konfigurací} nazveme trojici \mbox{\small{$(q, w, \alpha) \in Q \times \Sigma^* \times \Gamma^*$}}, kde $q$ je aktuální stav vnitřního řízení, $w$ je dosud nezpra\-covaná část vstupního řetězce a~$\alpha = Z_{i_1}Z_{i_2}\dots Z_{i_k}$ je obsah zásobníku\footnote{$Z_{i_1}$ je vrchol zásobníku}.
\end{definice}

\subsection{Podsekce obsahující větu a odkaz}
\begin{definice} \label{definice2}
Řetězec $w$ nad abecedou $\Sigma$ je přijat ZA \emph{A jestliže} ($q_0,w,Z_0)\underset{A}{\overset{*}{\vdash}}(q_F,\epsilon,\gamma)$ \emph{pro nějaké }$\gamma\in\Gamma^*$\emph{ a~} $q_F\in F$. \emph{Množinu }$L(A) = \{w\mid w$\emph{ je přijat ZA A}\}$\subseteq\Sigma^*$\emph{ nazýváme} jazyk přijímaný TS $M$.

\end{definice}
Nyní si vyzkoušíme vazbu vět a~důkazů opět s~použitím balíku \verb|amsthm|.

\begin{veta} \label{veta1}
\emph{Třída jazyků, které jsou přijímány ZA, odpovídá} bezkontextovým jazykům.
\end{veta}

\begin{proof}
V~důkaze vyjdeme z~Definice \ref{definice1} a~\ref{definice2}.
\end{proof}

\section{Rovnice a odkazy}
Složitější matematické formulace sázíme mimo plynulý text. Lze umístit několik výrazů na jeden řádek, ale pak je třeba tyto vhodně oddělit, například příkazem \verb|\quad|.

\[ 
\sqrt[i]{x^3_i} \quad \text{ kde } x_i \text{ je }i\text{-té sudé číslo splňující}\quad x_i^{2-x_i^{i^2}} \leq x_i^{y_i^3}
\]

V~rovnici (\ref{rovnice1}) jsou využity tři typy závorek s~různou explicitně definovanou velikostí.

\begin{eqnarray}
x &=& \bigg[\Big\{\big[a + b \big] * c \Big\}^d\ominus 1 \bigg]^{1/2} \label{rovnice1}\\
y &=& \displaystyle\lim_{x\to\infty}\frac{ \frac{1}{\log_{10} x}}{\textup{sin}^2 x + \textup{cos}^2 x} \nonumber
\end{eqnarray}

V~této větě vidíme, jak vypadá implicitní vysázení limity $\displaystyle \mathrm{lim}_{n\rightarrow\infty}\,f(n)$ v~normálním odstavci textu. Podobně je to i~s~dalšími symboly jako $\prod_{i=1}^n 2^i$ či $\bigcap_{A \in\mathcal{B}} A$. V~případě vzorců $\textstyle\lim\limits_{n\to\infty} f(n)$ a~$\underset{i=1}{\overset{n}{\prod}} 2^i$ jsme si vynutili méně úspornou sazbu příkazem \verb|\limits|.

\begin{eqnarray}
\int_b^a g(x)\, \mathrm{d}x &=&- \displaystyle \int\limits_a^b f(x)\, \mathrm{d}x\\
\overline{\overline{A \wedge B}} &\Leftrightarrow& \overline{\overline{A} \vee \overline{B}}
\end{eqnarray}

\section{Matice}
Pro sázení matic se velmi často používá prostředí \texttt{array} a~závorky (\verb|\left|, \verb|\right|).

\[
\left[ 
\begin{array}{ccc}
&\widehat{\beta + \gamma}&
\hat{\pi}\\
\vec{a}
&\overleftrightarrow{AC}&
\end{array}
\right]
= 1 \Longleftrightarrow \mathbb{Q} = \mathbf{R}
\]

\[
\textbf{A} = \left| 
\begin{array}{cccc}
a_{11} & a_{12} & \cdots & a_{1n}\\
a_{21} & a_{22} & \cdots & a_{2n}\\
\vdots & \vdots & \ddots & \vdots\\
a_{m1} & a_{m2} & \cdots & a_{mn}
\end{array}
\right|
=
\begin{array}{cc}
t&u\\
v&w\\
\end{array}
= tw-uv
\]

Prostředí \texttt{array} lze úspěšně využít i~jinde.

\[
\binom{n}{k} =
\left\{
\begin{array}{ll}
0 & \text{pro } k<0 \text{ nebo } k>n\\
\frac{n!}{k!(n-k)!} & \text{pro } 0 \leq k \leq n
\end{array}
\right.
\]

\end{document}